\section{Evaluation}\label{sec:evaluation}
As part of this work, the tool ESChecker~\cite{Reiser2016} has been implemented, applying the described algorithm. The set of implemented JavaScript features is not sufficient to perform an evaluation using realistic projects in broader practice. The evaluation uses sample listings to compare the implementation with Flow, TAJS, and Infernu\footnote{The evaluation uses the following versions: Flow 0.24.1, TAJS 0.9-7, and Infernu 0.0.0.1.}. Infernu and TAJS do not support ECMAScript 6 and therefore, ECMAScript 5 equivalents of the listings are used. 

The test cases and their results are summarized in \cref{tbl:evaluation-results}. Cells marked with a \checkmark~indicate that the test case (row) is handled correctly by the type checker (column). An empty cell indicates that the evaluation does not lead to the expected result. The test cases focus on features that are difficult to analyze statically --- as mentioned in \cref{sec:introduction} --- or features commonly used.


\begin{table*}
	\centering
	\ra{1.2}
	\begin{tabular}{@{}l c c c c@{}}\toprule
	Test Case & Infernu & TAJS & Flow & ESChecker \\ \midrule
	\nameref{ssec:variable-redefinement} &  & \checkmark & \checkmark & \checkmark \\
	\nameref{ssec:closures} & \checkmark & \checkmark & \checkmark & \checkmark \\
	\nameref{ssec:side-effects} & & \checkmark & & \checkmark \\ 
	\nameref{ssec:function-overloading} & & \checkmark & \checkmark & \checkmark \\ 
	\nameref{ssec:callbacks} & \checkmark & \checkmark &  & \checkmark \\ 
	\nameref{ssec:built-in-types} & & & \checkmark & \checkmark \\ 
	\nameref{ssec:dynamic-object-manipulation} & & \checkmark & & \\ 
	\nameref{ssec:dom-events} & & \checkmark & & \\ 
	Frameworks & & & & \\ \bottomrule
	\end{tabular}
	\caption{Evaluation Results}
	\label{tbl:evaluation-results}
\end{table*}

\subsection{Variable Redefinement}\label{ssec:variable-redefinement}
In JavaScript, values of different types can be assigned to the same variable. Therefore, the same variable can have different types at different positions in the program. The initialization of the variable is deferred in the following listing.

\javascriptfile{./evaluation/redefined-variable.js}

Infernu is the only tool that rejects the program --- that is well defined at runtime. The type system used by Infernu is over-restrictive. It prohibits value assignments to variables if the value type is not a subtype of the variable type. In this example, an object value is assigned to a variable of the incompatible type \textit{undefined}.

\subsection{Closures}\label{ssec:closures}
JavaScript supports closures, allowing functions to read and modify variables from the outer scope. The following listening defines the \textit{createUniqueIdGenerator} function that returns a generator function. The generator uses the \textit{currentId} variable as memory to remember the last returned value. This variable is only defined in the declaration scope of the generator but not in the caller's scope when the generator-function is invoked (line 8). Therefore, the analysis needs to be context-sensitive to support closures.

\javascriptfile{./evaluation/closure.js}

Closures are supported by all evaluated tools.

\subsection{Side Effects}\label{ssec:side-effects}
Functions in JavaScript do not have to be pure. A function call can have side effects to passed arguments or variables from the enclosing scope of the called function. For type checking, only side effects affecting the type of a variable are relevant. The function called in the following example adds the new property \textit{address} to the object passed as argument. This change of the arguments structure needs to be reflected in the caller's context.

\javascriptfile{./evaluation/effects.js}

Neither Flow nor Infernu allow side effects changing the structure of an argument and therefore reject the program. Adding new properties is intentionally prohibited by Flow to detect spelling errors.

This program is accepted by TAJS and ESChecker.

\subsection{Function Overloading}\label{ssec:function-overloading}
JavaScript does not provide built-in support for function overloading. A function can only have one definition. However, the number of arguments passed to a function does not have to exactly match the number of declared parameters. This allows optional arguments and therefore function overloading. The following listing defines and applies the \textit{range} function. The function can be called with one or up to three arguments\footnote{The implementation can not use the default parameters of ES6 as the semantic of the passed arguments depends on the number of arguments. If the function is called with a single argument, then the argument defines the \textit{end} of the range and not the \textit{start}. At the other hand, if \textit{range} is invoked with two arguments, then the first argument specifies the \textit{start} and the second the \textit{end}. Therefore, the semantic of an argument is not only defined by its position but also by the number of arguments.}. 

\begin{javascriptcode}
function range(start, end, step) {
	if (end === undefined) {
		end = start;
		start = 0;
	}
	
	if (step === undefined) {
		step = 1;
	}
	
	const result = ...;
	return result;
}

const r = range(10);
const r2 = range(1, 10);
const r3 = range(10, 1, -1);	
\end{javascriptcode}

Infernu does not support optional arguments and therefore rejects the program. The other tools support optional arguments.

\subsection{Callbacks}\label{ssec:callbacks}
Callbacks are commonly used in JavaScript for asynchronous and functional programing. Functions can be passed as values. This requires that the flow of function values is tracked. The following example implements the \textit{map} function. The \textit{map} function invokes a callback for every array element and puts the result in a new array that is returned. The \textit{map} function is a rank-1 polymorphic function, it can be applied with arrays of arbitrary types. Therefore, the type inference algorithm needs to infer the principle type and not a specialization~\cite{Pierce2002}. To verify that the principal type is inferred, the \textit{map} function is applied once with an array of numbers and once with an array of strings.

\begin{javascriptcode}
function map(array, mapper) {
	const result = [];
	// ... 
	return result;
}

const array = [1, 2, 3, 4, 5, 6];
const doubled = map(array, x => x * 2);

const names = ["Anisa", "Ardelia", "Madlyn"];
const lengths = map(names, name => name.length);
\end{javascriptcode}

Flow does not support type inference for rank-1 polymorphic functions. The example is accepted by Flow if either type annotations are added to \textit{map}, the function is only applied once, or the passed arrays are of equal types. All other tools support type checking of callbacks and type inference for rank-1 polymorphic functions.

\subsection{Built-in Types}\label{ssec:built-in-types}
JavaScript defines built-in objects and functions. These functions are natively implemented and not available as JavaScript source code, hence the source code can not be analyzed by the type inference algorithm. This requires that native object and functions are defined in the type checker or externally. The following example uses the native array functions \textit{map} and \textit{reduce}\footnote{As Infernu does not support optional arguments, all callback arguments have been added to the callbacks of \textit{map} and \textit{reduce} before analyzing the example with Infernu.}. 

The result of the \textit{reduce} function is an accumulated value over all array elements. The function uses a callback --- passed as first argument --- to accumulate the values in the array. The first argument of the callback is the accumulated value over all preceding array elements. The second argument of the callback is the current array element. The callback adds the current element to the accumulated value of the preceding elements and returns the accumulated value. The second argument of \textit{reduce} is the initial accumulator value. The \textit{reduce} function is invoked with the arguments in incorrect order to verify if the type checker validates callbacks passed to built-in functions.

\javascriptfile{./evaluation/built-ins.js}

Infernu rejects the program as it does not support the built-in \textit{reduce} method. TAJS rejects the program as it does not support the built-in \textit{map} method. Flow and ESChecker correctly type the application of the \textit{map} function and state the incorrect application of the \textit{reduce} function.

\subsection{Dynamic Object Manipulation}\label{ssec:dynamic-object-manipulation}
The structure of an object can be dynamically defined or manipulated by using reflection-like code. Such code uses computed property names. The following listing shows an implementation of the widely used \textit{defaults} function. The \textit{defaults} function accepts two objects and initializes the properties with value \textit{undefined} of the \textit{target} object with the values defined in \textit{source}. This pattern is commonly used to initialize absent properties with default values for option-objects.

\javascriptfile{./evaluation/dynamic-object-manipulation.js}
A precise and sound analysis needs to unroll the \textit{for} loop (line 3) to know which properties are copied from the \textit{source} to the \textit{target} object (line 4). Unrolling requires that the analysis understands the semantics of \textit{Object.keys}. 

This example initializes an empty object with default values (line 10). The initialized option-object is used on line 11, but the property name \textit{rnds} --- for \textit{rounds} --- is misspelled. TAJS is the only implementation that detects the misspelled property name, but only if a \textit{for in} loop is used instead of the unsupported \textit{Object.keys} method. But it also emits a warning that the property \textit{step} is potentially absent, and thus a false positive. Flow and ESChecker are unsound for such dynamic code and therefore do not detect the misspelled property name. Infernu does not support \textit{for in} loops and therefore fails to type check the program.

\subsection{DOM Events}\label{ssec:dom-events}
JavaScript is often used in web applications to add dynamic effects by using the DOM-API. A type checker needs to model the DOM-API, as needed for other built-in types like arrays. Modeling the types and methods defined by the DOM-API is insufficient to achieve type safety, the type checker also needs to model the DOM events~\cite{JensenMadsenMoller2011}. 

A JavaScript application reacts to user input by registering a listener for a particular event on a specific DOM element, like an input field. The API for registering listeners is generic, but the event passed to the listener depends on the type of the handled event. A \textit{keydown} event passes a \textit{KeyboardEvent} object to the listener. The \textit{KeyboardEvent} has additional properties allowing the identification of the pressed key. The type of the event is defined by the event name specified when the listener is registered as shown in the following listing on line 10 and 11.

\javascriptfile{./evaluation/event-handlers.js}

Registering the \textit{onKeyDown} listener for the \textit{keydown} event is legitimate as the \textit{KeyboardEvent} defines the \textit{getModifierState} method used by the listener. On the other hand, registering the same listener for the \textit{blur} event is erroneous as the \textit{blur} event does not define the \textit{getModifierState} method. 

TAJS is the only tool that detects the malicious invocation of the \textit{getModifierState} method in case of a \textit{blur} event. But TAJS also reports a warning that the variable \textit{input} can potentially be null on line 10 and 11, regardless of the preceding null check (line 9). ESChecker also reports an error that \textit{input} is potentially absent. The reason for this is that ESChecker is neither path-sensitive nor does it evaluate the test condition. Therefore, \textit{input} is potentially absent in all branches and accessing a property is wrongly detected as error for all branches.

Infernu fails to type check the given example as it does not model the DOM-API. Flow does not reject the given program but neither detects the invocation of the not defined function \textit{getModifierState} for the \textit{blur} event. 

A special handling for DOM events is beneficial, as it adds additional type safety. But it is noteworthy that events are not only used when interacting with the DOM. Events are also used by frameworks like Angular~\cite{Angular} or Ember~\cite{Ember} internally or as part of their API. A type checker therefore does not only need to model the DOM events but also the framework events to achieve type safety.

\subsection{The Impact of Frameworks on Type Inference}
Frameworks are a special challenge for type inference as they make heavy use of dynamic invocations.  Frameworks are implemented according to the inversion of control principle, the framework invokes the application, not vice versa. This inverses the control- and data-flow, that are decisive for type inference. 

Inferring the correct type is often only possible if the types of the arguments are known. Without an actual invocation, these are unknown. The Angular~\cite{Angular} controller implementation shown in this example loads a remote resource using the \textit{\$http} service (line 8) and stores the result in the \textit{persons} array (line 9). The name of the \textit{get} method is misspelled. 

\javascriptfile{./evaluation/angular-controller.js}

The problem shown by this example is that type inference for framework-based code --- if the framework uses dynamic invocation --- is  hardly possible for a framework-agnostic type checker, because the control and data-flow can not be inferred using static analysis. The \textit{\$http} service accepted in the constructor (line 2) is a service that is injected by angular at runtime. The implementation to inject is resolved by name, therefore no explicit connection between the service usage and the service implementation exists. But the type of \textit{\$http} must be known to perform type checking. As the type for \textit{\$http} is not known, its usages can not be type checked. Explicit type annotations are needed for the \textit{\$http} parameter in the constructor. 

\subsection{Interpretation of Evaluation Results}
The evaluation shows that none of the evaluated tools supports all test cases. ESChecker achieves preciser results than Flow and Infernu, but not as precise as TAJS. TAJS has the most precise type inference and therefore achieves the best type checking results. TAJS preliminary limitation is the lack of support for many ECMAScript 5 features. This might be the major impediment for its adaption for real world projects. It also tends to emit false positives if the precision of the type inference diminishes. Flow at contrary has the best support for ECMAScript 6 features but type inference is less precise. The precision can be increased by adding type annotations. The type system used by Infernu is over-restrictive and rejects most of the test cases, all of them are well typed at runtime. Compared to Infernu, ESChecker has a lower false positive rate and therefore combining the Hindley Milner Algorithm W with data-flow analysis has proven to be beneficial to increase the precision. 

But the precision is not sufficient for all examples. Path-sensitivity is needed to access members of a potentially \textit{null} object, otherwise the value is potentially null in all branches, independent of preceding null checks. Path-sensitivity is also needed for type-specific branches, commonly used for function overloading.

The evaluation also indicates that the precision for the type inference diminishes for reflection-like code, for instance, if the structure of an object is dynamically manipulated. A considerable alternative to type inference for these edge cases --- supporting this edge cases might add a high complexity to the type inference algorithm --- is the use of type annotations to improve the overall result. Type annotations can either be extracted from JSDoc or be defined in external declaration files, similar to the TypeScript definition files. This increases the precision of type inference and therefore leads to better type checking results. A tool implementing this approach should emit a warning if the type inference algorithm can not infer the types to encourage the programmer to add type annotations. Allowing any form of type annotations also has the benefit that functions invoked by frameworks can be annotated and therefore can be type checked.
