\section{Conclusion}\label{sec:conclusion}
The tool support for JavaScript is especially small compared to its popularity. Developers need to relay on manual testing or unit tests for revealing programming errors. The implemented, unsound type checker provides a tool that is capable to infer the types and catch a variety of errors through type checking. The evaluation showed that the analysis is precise for most of the scenarios and sometime provides better results then competitive tools. Therefore, the tool can provide valuable feedback. But the evaluation also shows that the precision abruptly decreases when objects are dynamically manipulated. 

The use of dynamic object manipulation only affects type inference and not the type checking algorithm. A considerable alternative to adding support for these edge cases --- that might add a high complexity to the type inference algorithm --- is the used of type annotations to improve the overall result. Type annotations can either be extracted from JSDoc or be defined in external declaration files, similar to the TypeScript definition files. This increases the precision of the type inference algorithm and therefore the achievable results with type checking. A tool implementing this approach should emit a warning if the type inference algorithm cannot infer the types to motivate the programmer to annotate these terms. Allowing any form of type annotations also allows type checking of functions invoked by frameworks that otherwise cannot be analyzed. 

The evaluation also showed that the used algorithm has its limitations. As the analysis is not path sensitive, possible null values cannot be accessed as their value is potentially always null in all branches. Another issue is that --- without path sensitivity --- type specific branches cannot be implemented, a technique often used to emulate function overloading. 

Further the set of supported features is not sufficient to analyze real word projects. The implementation is still missing elementary features like classes, prototyping or modules, these are all essential features needed before the tool is useful.  Supporting the features defined in ES6 and in the upcoming ES7 standard requires a tremendous amount of additional work that exceeds the scope of a project thesis by a multitude. But this project thesis showed that precise and good type inference results can be achieved for a majority of JavaScript that, combined with type checking, provides an immediate feedback to the developer. The cases where effective and precise type inference is not achievable can be substituted by adding type annotations. The analysis might not be sound, but the gained support is worth it.